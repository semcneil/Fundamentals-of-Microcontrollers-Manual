\chapter{DC Motors and Control}
\chaplabel{dcMotors}

\section{Introduction}
This chapter introduces students to types of DC motors and controlling DC motors using H-bridge type devices.

\section{Types of DC Motors}
The types of DC motors relevant to this class are as follows:
\begin{enumerate}
	\item Brushed
	\begin{enumerate}
		\item DC 
		\item Hobby Servos
	\end{enumerate}
	\item Brushless
	\item Stepper
\end{enumerate}

\subsection{Brushed DC}
Brushed DC motors are very common. The haptic motors that make your phone vibrate are likely brushed DC motors.
The motors in most toys are also brushed DC motors. They are cheap to make and easy to use so they are very
common. The direction of rotation is controlled by changing the polarity of the applied voltage. The speed of 
rotation is controlled by varying the magnitude of the applied voltage. The torque of a brushed DC motor 
increases with rotational speed. The advantages and disadvantages of brushed DC motors is outlined in 
Table \ref{table:dcbrushed}.

\begin{table}[!ht]
	\centering
	\begin{tabular}{l l}
		\hline
		Advantages & Disadvantages \\ 
		\hline
		Inexpensive & Mechanical noise from brushes \\
		Lightweight & Electrical noise from brushes \\
		Reasonably efficient  & \\
		\hline
	\end{tabular}
	\caption{Advantages and disadvantages of brushed DC motors.}
	\label{table:dcbrushed}
\end{table}

The brushes on brushed DC motors change which coil in the rotor is activated as the rotor rotates such 
that the rotor is always pushing away from the permanent magnets in the stator. The brushes transfer 
the current from the stator to the rotor by having electrical brushes contacting metal patches on the 
rotor.

\subsection{Hobby Servos}

\subsection{Brushless DC Motors}
Brushless DC motors have the coils in the stator and the permanent magnets on the rotor. The coils are 
activated in sequence to keep the rotor spinning. An electronic speed controller (ESC) controls the 
coil activation to keep the motor spinning at the desired rate. Some ESCs make use of a sensor on the 
motor to tell which coil needs activating to keep the motor spinning. Sensorless ESCs (common in the 
hobby market) measure the back emf (voltage across each coil) to know when to activate each coil.

It is important to choose a good quality ESC that is rated sufficiently to drive the motor. I have had 
a couple catastrophic failures of ESCs in flight. Fortunately, they were on fixed wing drones with 
good pilots so there was no other loss of payload/aircraft. 

Brushless DC motors are used all around as well in things like computer fans, hard drive platter spinners,
 drones, and hybrid vehicles. The advantages and disadvantages of brushless DC motors are outlined in 
 Table \ref{table:dcbrushless}.

 \begin{table}[!ht]
	\centering
	\begin{tabular}{l l}
		\hline
		Advantages & Disadvantages \\ 
		\hline
		Quiet & Usually require separate ESC \\
		Efficient & \\
		\hline
	\end{tabular}
	\caption{Advantages and disadvantages of brushless DC motors.}
	\label{table:dcbrushless}
\end{table}

\subsection{Stepper Motors}
Stepper motors move one step at a time which makes them very useful in situations where fine motion control 
is needed. Position control is possible without any feedback mechanism when using stepper motors. However, 
if the motor is under too large a load, it may skip a step and position estimation will be off. Stepper 
motors have highest torque at low speed with torque dropping as speed increases. They require at least 
two H-bridges to drive. 

\section{References}
\begin{enumerate}
	\item \href{https://learn.adafruit.com/adafruit-motor-selection-guide?view=all}{Adafruit Motor Selection Guide}
	\item \href{https://learn.sparkfun.com/tutorials/hobby-servo-tutorial/all}{SparkFun Servo Tutorial}
	\item \href{https://www.arduino.cc/reference/en/libraries/servo/}{Arduino Servo Library}
	\item \href{https://www.ti.com/lit/an/slva767a/slva767a.pdf}{TI Stepper Motor Reference}
\end{enumerate}