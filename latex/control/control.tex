\chapter{Control of Systems}
\chaplabel{control}

\section{Introduction}
This chapter introduces students to some basic control strategies including PID.


\section{PID Control}
The parallel (how we usually draw the controller) form of the PID 
equation is shown in Equation \ref{eq:pidpar}.
\begin{equation}
    \label{eq:pidpar}
    u(t) = K_p e(t) + K_i\int_0^t e(\tau)d\tau + K_d\frac{d}{dt}e(t)
\end{equation}

The standard form of the PID equation rearranges the gains so that they have
a more easily understood physical meaning as shown in Equation \ref{eq:pidstd}.

\begin{equation}
    \label{eq:pidstd}
    u(t) = K_p\left(e(t) + \frac{1}{T_i}\int_0^t e(\tau)d\tau + T_d\frac{d}{dt}e(t)\right)
\end{equation}

In the standard form, $T_i$ is the time it will take to eliminate all errors
assuming the loop control does not change. $T_d$ is how far into the future
the derivative term is trying to predict the error. Note that times in this 
context can either be in seconds or samples. Samples is more typical in an 
actual implementation.

\subsection{Proportional Control}
Sometimes just using the proportional term is enough for controlling a system.
The equation is shown in Equation \ref{eq:pidpro}.

\begin{equation}
    \label{eq:pidpro}
    u(t) = K_p e(t)    
\end{equation}

\subsection{Integral Control}
It is a rare system that only requires integral control but the equation 
for integral control is shown in Equation \ref{eq:pidint}.
\begin{equation}
    \label{eq:pidint}
    u(t) = K_i\int_0^t e(\tau)d\tau   
\end{equation}

\subsection{Differential Control}
It is also rare that a system can be controlled satisfactorily with only
differential control, but for completeness it is shown in Equation \ref{eq:piddiff}.
\begin{equation}
    \label{eq:piddiff}
    u(t) = K_d\frac{d}{dt}e(t) 
\end{equation}

\subsection{PI and PD Control}
PI and PD control are sometimes sufficient to control a system satisfactorily.

