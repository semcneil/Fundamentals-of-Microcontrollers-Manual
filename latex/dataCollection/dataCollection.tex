\chapter{Data Collection and Environmental Sensing}
\chaplabel{data}

\section{Introduction}
This chapter introduces students to the concepts of collecting data in general and environmental data in particular.


\section{Laboratory Exercises}
\subsection{To Do}
For the lab, collect the following data and display it once a second on the display with the 
appropriate units if you can make them fit.
\begin{enumerate}
	\item The output from \lstinline@millis()@ (ms)
	\item Light intensity as a number between 0 and 1023 (unitless)
	\item Potentiometer value as a voltage between 0 and 3.3~V
	\item Battery voltage as a voltage between 0 and 6.6~V
	\item Temperature in Fahrenheit
	\item Relative humidity in \%
	\item Distance as a number (unitless)
	\item Color as integers between 0 and 255 in the form (R,G,B) (unitless)
	\item Accelerometer readings in g's (ax, ay, az) (g)
	\item Gyroscope readings in degrees per second (gx, gy, gz) (dps)
\end{enumerate}

\subsection{Suggestions}
Do the temperature and humidity last since the SHT31 sensor sometimes requires power cycling 
to get working after uploading code to the Nano Connect.

Use a \href{https://www.arduino.cc/reference/en/language/variables/data-types/stringobject/}{String} 
object to accumulate your display string and then call \lstinline@display.println(yourString)@ 
to display it. Note a few things:
\begin{enumerate}
	\item The String type starts with a capital S.
	\item You can add to the String object using \lstinline@+=@ or just \lstinline@+@, 
		but with only the plus operator, all arguments have to be of the same type. 
	\item The String object also allows you to limit the number of decimal places for \lstinline@float@ types. 
\end{enumerate}
An example is shown in Listing \ref{lst:dispstr}.
\begin{lstlisting}[caption={This is an example of using a String 
		object to display text and float variables. The floats are 
		limited to 1 decimal place such that 7.123 would be displayed as 7.1.},
		label={lst:dispstr},language=C++]
	String dispStr;
	dispStr = "T(F),H(%): ";
	dispStr += String(tF,1);
	dispStr += ",";
	dispStr += String(humidity,1);
	dispStr += "\n";
	// Control the display  
	display.clearDisplay();
	display.setTextSize(1);  // Normal 1:1 pixel scale
	display.setCursor(0,0);  // Start at top-left corner
	display.println(dispStr);
	display.display();
\end{lstlisting}

\subsection{Turn In}
Submit a copy of your code and a video (or link to a video) of your board running your code. The 
video should include both lab partners faces.