\chapter{Inertial Measurements}
\chaplabel{imu}

\section{Introduction}
This chapter introduces students to using collecting inertial data such as
linear and angular acceleration.

\section{Rectilinear Kinematics}
Rectilinear kinematics is about motion along a straight line. This provides a good starting point for 
discussing the mathematics of inertial measurements. 

Time, position, velocity, and acceleration have the following differential relationships:
\begin{subequations}
	\label{eq:rectkin}
	\begin{align}
		a = & \dv{v}{t} \\
		v = & \dv{s}{t} \\
		a \ \dd s = & v\  \dd v
	\end{align}
	
\end{subequations}

If the acceleration is known (or can be assumed to be) constant, Equations \ref{eq:rectkin} can be
integrated to give Equations \ref{eq:constakin}.

\begin{subequations}
	\label{eq:constakin}
	\begin{align}
		v = & v_0 + a_ct \\
		s = & s_0 + v_0t + 0.5a_ct^2 \\
		v^2 = & v_0^2 + 2a_c(s - s_0)
	\end{align}
\end{subequations}

Constant acceleration is usually applied to the kinematics of projectiles where the constant acceleration
is due to gravity. In the case of a digital IMU, the acceleration measurement is reported periodically 
(at 104~Hz in the default case of the LSM6DSOXTR on the Nano RP2040 Connect). Since it is a sampled system,
we cannot just integrate Equations \ref{eq:rectkin} to get velocity and position. What we do instead is to
assume constant acceleration between samples and run a cumulative sum to calculate the velocity and position.

Integrating also assumes the knowledge of initial conditions. Usually we start with an initial condition of 
being at rest. This simplifies our starting point. At each subsequent calculation, the output of the previous
sample is taken as the initial condition. 

